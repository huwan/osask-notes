\chapter{	分割编译与中断处理	}
\section{	分割源文件(harib03a)	}
将bootpack.c分割成了graphic.c、dsctbl.c、bootpack.c。
\section{	整理Makefile(harib03b)	}
将处理相同事情的部分归纳,按照一般规则来处理。
\section{	整理头文件(harib03c)	}
将各个源文件重复部分去掉,归纳起来,放到bootpack.h的头文件里。在具体的源文件里引用头文件。
\section{	意犹未尽	}
介绍上一章naskfunc.nas的|load_gdtr|函数:
\begin{code}
_load_gdtr:		; void load_gdtr(int limit, int addr);
		MOV		AX,[ESP+4]		; limit
		MOV		[ESP+6],AX
		LGDT	[ESP+6]
		RET
\end{code}
函数用来将指定的段上限(limit)和地址赋值给名为GDTR的48位寄存器。这是一个很特别的48位寄存器,并不能用我们常用的MOV指令来赋值。给它赋值的时候,唯一的方法就是指定内存地址,从指定的地址读取6个字节,然后复制给GDTR寄存器。完成这一任务的就是LGDT。

该寄存器的低16位是段上限,它等于“GDT的有效字节数-1”。剩下的高32位代表GDT的开始地址。

在最初执行这个函数的时候,|DWORD[ESP+4]|里存放的是段上限,|DWORD[ESP+8]|里存放的是地址。具体到实际的数值,就是0x000fff和0x00270000。把它们按字节写出来的话就成了|FF FF 00 00 00 27 00|(注意地位放在内存地址小的字节里)。为了执行LGDT,希望把它们排列成|FF FF 00 00 00 27 00|的样子,所以就先用“|MOV AX,[ESP+4]|”读取最初的0xffff,然后再写到|[ESP+6]|里。这样结果就成了|[FF FF FF FF 00 27 00 00]|,如果从|[ESP+6]|开始读6字节的话,正好是我们想要的结果。

书中补充了dsctbl.c里的|set_segmdesc|函数及相关段的知识。具体内容参见书本。
\section{	初始化PIC(harib03d)	}
为了达到鼠标指针移动的目的,必须使用中断,而要使用中断必须将GDT和IDT正确无误的初始化。此时,需要初始化PIC。

PIC(programmable interrupt controller),可编程中断控制器。PIC将8个中断信号(IRQ)集合成一个中断信号的装置。PIC监视输入管脚的8 个中断信号,只要有一个中断信号进来,就将唯一的输出管脚信号变成ON,并通知给CPU。最初设计者希望通过增加PIC来处理更多的中断信号,把中断信号设计成15 个,增设了2个PIC(主从PIC)。

\cs

\dag|projects\06_day\harib03d|
\begin{code}[label=int.c的主要组成部分]
void init_pic(void)
/* PICの初期化 */
{
	io_out8(PIC0_IMR,  0xff  ); /* 全ての割り込みを受け付けない */
	io_out8(PIC1_IMR,  0xff  ); /* 全ての割り込みを受け付けない */

	io_out8(PIC0_ICW1, 0x11  ); /* エッジトリガモード */
	io_out8(PIC0_ICW2, 0x20  ); /* IRQ0-7は、INT20-27で受ける */
	io_out8(PIC0_ICW3, 1 << 2); /* PIC1はIRQ2にて接続 */
	io_out8(PIC0_ICW4, 0x01  ); /* ノンバッファモード */

	io_out8(PIC1_ICW1, 0x11  ); /* エッジトリガモード */
	io_out8(PIC1_ICW2, 0x28  ); /* IRQ8-15は、INT28-2fで受ける */
	io_out8(PIC1_ICW3, 2     ); /* PIC1はIRQ2にて接続 */
	io_out8(PIC1_ICW4, 0x01  ); /* ノンバッファモード */

	io_out8(PIC0_IMR,  0xfb  ); /* 11111011 PIC1以外は全て禁止 */
	io_out8(PIC1_IMR,  0xff  ); /* 11111111 全ての割り込みを受け付けない */

	return;
}
\end{code}

以上是PIC的初始化程序。从CPU的角度看,PIC是外部设备,CPU使用OUT指令进行操作,程序中的PIC0和PIC1,分别指主PIC和从PIC。PIC内部有很多寄存器,用端口号码对彼此进行区别,以决定是写入哪一个寄存器。具体的端口号码写在bootpack.h里。

\cs

PIC寄存器是8位寄存器。IMR(interrupt mask register)是中断屏蔽寄存器。8位分别对应8路IRQ信号。如果某一位的值是1,则该位所对应的IRQ信号被屏蔽,PIC就忽略该路信号。这主要是因为,正在对中断设定进行更改时,如果再接受别的中断信号会引起混乱,为了防止这种情况发生,就必须屏蔽中断。还有,如果某个IRQ没有连接任何设备的话,静电干扰也可能会引起反应,导致操作系统混乱,所以也要屏蔽掉这类干扰。

ICW(initial control word)是初始化控制数据。ICW有4个,分别编号为1$\sim$4,共有4个字节的数据。ICW1和ICW4主板配线方式、中断信号的电气特性等有关,不再叙述。ICW3是有关主-从连接的设定,对主PIC而言,第几号IRQ与从PIC相连,是用8位来设定的。对从PIC而言,该PIC与主PIC的第几号相连是用3位来设定。

\cs

不同的操作系统可以进行独特设定的只有ICW2。它决定了IRQ以哪一个信号中断通知CPU。

这次是以|INT 0x20~0x2f|接收中断信号IRQ0$\sim$15而设定的,因为INT 0x00$\sim$0x1f用于应用程序想对操作系统干坏事的时候CPU内部系统保护通知,所以直接使用|INT 0x20~0x2f|。

\section[中断处理程序的制作(harib03e)]{中断处理程序的制作(harib03e)\protect\footnote{书中讲到了IRQ1和IRQ12的中断处理程序,光盘里面补充了IRQ7的中断处理程序。详细解释见书。}}
鼠标是IRQ12,键盘是IRQ1,编写了用于INT 0x2c和INT 0x21的中断处理程序(handler),即中断发生时要调用的程序。

\begin{code}[label=int.c节选]
void inthandler21(int *esp)
/* PS/2キーボードからの割り込み */
{
	struct BOOTINFO *binfo = (struct BOOTINFO *) ADR_BOOTINFO;
	boxfill8(binfo->vram, binfo->scrnx, COL8_000000, 0, 0, 32 * 8 - 1, 15);
	putfonts8_asc(binfo->vram, binfo->scrnx, 0, 0, COL8_FFFFFF, "INT 21 (IRQ-1) : PS/2 keyboard");
	for (;;) {
		io_hlt();
	}
}
\end{code}
程序只是显示一条信息,然后保持在待机状态。鼠标的程序也几乎完全相同。

\cs

中断执行后不能执行return,而是必须执行IRETD指令。借助汇编语言修改naskfunc.nas。

\dag|projects\06_day\harib03e|
\begin{code}[label=naskfunc.nas节选]
		EXTERN	_inthandler21, _inthandler27, _inthandler2c

_asm_inthandler21:
		PUSH	ES
		PUSH	DS
		PUSHAD
		MOV		EAX,ESP
		PUSH	EAX
		MOV		AX,SS
		MOV		DS,AX
		MOV		ES,AX
		CALL	_inthandler21
		POP		EAX
		POPAD
		POP		DS
		POP		ES
		IRETD
\end{code}

其中,PUSHAD 相当于:
\begin{code}
PUSH EAX
PUSH ECX
PUSH EDX
PUSH EBX
PUSH ESP
PUSH EBP
PUSH ESI
PUSH EDI
\end{code}
POPAD相当于按以上相反的顺序,把它们全都POP出来。

\cs

函数只是将寄存器的值保存到栈里,然后将DS和ES调整到与SS相等,再调用\_inthandler21,返回以后,将所有寄存器的值再返回到原来的值,然后执行IRETD。

\cs

将这个函数注册到IDT中去,在dsctbl.c的|init_gdtidt|里加入以下语句。

\begin{code}
    /* IDTの設定 */
	set_gatedesc(idt + 0x21, (int) asm_inthandler21, 2 * 8, AR_INTGATE32);
	set_gatedesc(idt + 0x27, (int) asm_inthandler27, 2 * 8, AR_INTGATE32);
	set_gatedesc(idt + 0x2c, (int) asm_inthandler2c, 2 * 8, AR_INTGATE32);
\end{code}
|asm_inthandler21|注册在idt的第0x21号。这里的2$\times$8表示的是|asm_inthandler21|属于哪一个段,即段号是2,乘以8是因为低三位有别的意思,这里低三位必须为0,相当于写成“2<<3”。

号码为2的段,正好涵盖了整个bootpack.hrb。最后的|AR_CODE32_ER|将IDT的属性设定为0x008e。它表示这是用于中断处理的有效设定。
\begin{code}
set_segmdesc(gdt + 2, LIMIT_BOTPAK, ADR_BOTPAK, AR_CODE32_ER);
\end{code}

\cs

对bootpack.c的Harimain的补充。“|io_sti();|”仅仅是执行STI指令,它是CLI的逆指令。在HariMain的最后,修改了PIC的IMR,以便接收来自键盘和鼠标的中断。

\cs

运行程序会发现,键盘的中断响应正常,鼠标的却不行。










