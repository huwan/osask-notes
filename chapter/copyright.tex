\begin{publishcopyright}

本文是在我的毕业设计指导老师邱德慧老师的悉心指导下完成的。衷心感谢邱老师在过去
的四年中对我的学习和科研生活所倾注的大量时间和心血!特别是从大学生创新项目、专业实习开始,一直到整个毕设设计的全过程,邱老师不厌其烦地教导我如何用科学的方法做科研,并毫无保留的与我分享她的科研经验,这对于我来说是一笔非常宝贵的财富。

邱老师治学严谨,无论是从汇编语言,EDA/SOPC课程的教学中细致的教学准备,还是在大学生创新项目中对学生的悉心指导中都可以看出来,也让我从中学到了踏实严谨的学习态度。毕业设计的整个完成过程并不是一番风顺的,由于自己经常被被琐事羁绊,曾一度导致毕业设计的进度停滞不前,是邱老师坦诚地谈话、严厉地批评和不断地鼓励让我坚持下来,认真完成毕业设计。邱老师给我提供了很多锻炼自己的机会,在学习上也是一如既往地支持我,这一切都令我感激不已。很幸运能够在邱老师的指导下完成我的毕业设计。

在论文研究过程中,同样也得到了其他同学和老师的帮助,感谢同小组的周邦、贾丽君、刘旭等同学对我的督促和帮助,在此向他们表达诚挚的谢意。


 “为学为师,求实求新”,从第一次看到欧阳中石老先生书写在校本部主楼前的校训,到现在已经度过了大学四年的大部分时光。四年的大学学习生活即将结束,在过去
 的四年中,我还得到了很多老师同学的帮助,在此,我要向所有帮助过我、关心过我的老师、同学和朋友们致以最诚挚的谢意!


\end{publishcopyright}
