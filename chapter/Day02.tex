\chapter{	汇编语言学习与Makefile入门	}
\section{	介绍文本编辑器	}
如中文译者所注,推荐使用Notepad++文本编辑器。

使用这个软件打开光盘中提供的源代码会出现日语注释乱码,选择~格式-->编码字符集--> 日文-->Shift-JIS~即可正常显示代码中原书作者的日语注释。但是,关闭文件之后重新打开又会恢复原状,解决方法为在选择Shift-JIS编码后复制内容到一个新的文件中去,保存替换原先的文件即可,需要提示的是,新建的文件要使用ANSI编码格式保存,不能是UTF-8,否则会导致后面编译时报错。

原书作者推荐的文本编辑器TeraPad在中文系统中会出现软件本身的文本乱码,如菜单栏工具栏的文字,但是无需设置就可以正常显示代码中的日语注释。

\section{	继续开发	}
选讲程序的核心部分,核心程序之前和启动区以外的内容需要具备软盘方面的相关知识,后面讲。
\dag |projects\02_day\helloos3|
\begin{code}[label=helloos.nas]
; hello-os
; TAB=4

		ORG		0x7c00			; 指明程序的装载地址

; 以下的记述用于标准FAT12格式软盘

		JMP		entry
		DB		0x90
---中略---
; 程序核心

entry:
		MOV		AX,0			; 初始化寄存器
		MOV		SS,AX
		MOV		SP,0x7c00
		MOV		DS,AX
		MOV		ES,AX

		MOV		SI,msg
putloop:
		MOV		AL,[SI]
		ADD		SI,1			; 给SI加1
		CMP		AL,0
		JE		fin
		MOV		AH,0x0e			;显示一个文字
		MOV		BX,15			; 指定字符颜色
		INT		0x10			; 调用显卡BIOS
		JMP		putloop
fin:
		HLT						; 让CPU停止,等待指令
		JMP		fin				; 无限循环

msg:
		DB		0x0a, 0x0a		; 换行两次
		DB		"hello, world"
		DB		0x0a			; 换行
		DB		0
\end{code}
\cs

ORG指令:程序从指定的这个地址开始,也就是把程序装载到内存中的指定地址。这里是0x7c00。

JMP指令:无条件跳转。配合下面的标签"entry"等,可指定跳转的目的地。

“entry”:标签声明,用于指定JMP指令的跳转地址。

MOV指令:赋值。“MOV AX,0”相当于“AX=0”这样一个赋值语句。同样,“MOV SS,AX”相当于“SS=AX”。
\cs

相关寄存器:

16位寄存器:AX、CX、DX、BX、SP、BP、SI、DI;

其中,前四个的高低8位可当8位寄存器用,AL、CL、DL、BL、AH、CH、DH、BH;

32位寄存器:16位寄存器可以扩展为32位寄存器,EAX、ECX、EDX、EBX、ESP、EBP、ESI、EDI;

段寄存器:ES、CS、SS、DS、FS、GS;
\cs

“MOV SI,msg”:这里是可以“将标号赋值给寄存器”。在汇编语言中,所有的标号都仅仅是单纯的数字。每个标号对应的数字,是由汇编器根据ORG指令计算出来的。编译器计算出的“标号的地方对应的内存地址”就是那个标号的值。这里,msg的地址是0x7c74,所以这个指令就是把0x7c74带入到SI寄存器去。

“MOV AL,[SI]”:MOV指令的数据传送源和传送目的地不仅可以是寄存器或常数,也可以是内存地址。这个时候,我们使用方括号(|[]|)来表示内存地址。另外,可以用来指定内存地址的寄存器只有BX、BP、SI、DI这几个。

可以用下面指令将SI地址的1字节内容读入到AL:

|MOV AL, BYTE [SI]|

由于MOV指令的规则,即源数据和目的数据必须位数相同,也就是向AL里代入的只能是BYTE,这样以来就可以省略BYTE,即可以写成:

|MOV AL, [SI]|
\cs

ADD指令是加法指令。若以C语言的形式改写“ADD SI,1”的话,就是“SI=SI=1”。

CMP是比较指令。“CMP AL,0”,是将AL中的值与0进行比较。

JE 是条件跳转指令之一。所谓的条件跳转指令,就是根据比较的结果决定跳转或者不跳转。就JE指令而言,如果比较结果相等,则跳转到指定的地址;而如果比较结果不等,则不跳转,继续执行下一条指令。

|CMP AL, 0|

|JE fin|

这两条指令相当于:

|if(AL == 0){ goto fin;}|
\cs

INT是软件中断指令。在这里是调用显卡 BIOS,不解释。
\cs

HLT指令,让CPU停止动作,进入待机状态。
\cs

用C语言改写helloos.nas程序节选。
\begin{code}
entry:
	AX = 0;
	SS = AX;
	SP = 0x7c00;
	DS = AX;
	ES = AX;
	SI = msg;
putloop:
	AL = BYTE [SI];
	SI = SI+1;
	if (AL ==0 ){goto fin;}
	AH = 0x0e;
	BX = 15;
	INT 0x10;
	goto putloop;
fin:
	HLT;
	goto fin;
\end{code}

就是有了这个程序,我们才能把msg里写的数据,一个字符一个字符地显示出来,并且数据变成0以后,HLT指令就会让程序进入无限循环,“hello,world” 就是这样显示出来的。

\cs

程序中的ORG 后地址0x7c00是因为目前约定内存的|0x00007c00-0x00007dff|地址为启动区内容的装载地址,不能随便改成其它地址。

\section{	先制作启动区	}

\begin{quote}
考虑到以后的开发,不要一下子用nask来制作整个磁盘映像,而是先用它来制作512字节的启动区,剩下的部分我们用磁盘映像管理工具来做。

先把helloos.nas的后半部分截掉,这是因为启动区只需要最后的512字节。现在这个程序仅仅用来制作启动区,所以把文件名改为ipl.nas。

然后改造asm.bat,将输出的文件名改成ipl.bin。另外,也顺便输出列表文件ipl.lst。这是一个文本文件,可以用来简单地确认每个指令是怎么翻译成机器语言的。

另外还增加了一个makeimg.bat文件。它是以ipl.bin为基础,制作磁盘映像文件helloos.img 的批处理文件。利用作者开发的磁盘映像管理工具edimg.exe,先读入一个空白的磁盘映像文件,然后在开头写入ipl.bin,最后输出名为helloos.img的磁盘映像文件。

这样,从编译到测试的步骤为双击!cons,然后在命令行窗口中按顺序输入asm -->makeimg -->run这3个命令。
\end{quote}

下一节有更简单的编译方式。


\section{	Makefile入门	}

作者编写了一个Makefile文件,这样就可以方便的生成所需要的文件。

\dag |projects\02_day\helloos5|
\begin{code}[label= Makefile]

# デフォルト動作

default :
	../z_tools/make.exe img

# ファイル生成規則

ipl.bin : ipl.nas Makefile
	../z_tools/nask.exe ipl.nas ipl.bin ipl.lst

helloos.img : ipl.bin Makefile
	../z_tools/edimg.exe   imgin:../z_tools/fdimg0at.tek \
		wbinimg src:ipl.bin len:512 from:0 to:0   imgout:helloos.img

# コマンド

asm :
	../z_tools/make.exe -r ipl.bin

img :
	../z_tools/make.exe -r helloos.img

run :
	../z_tools/make.exe img
	copy helloos.img ..\z_tools\qemu\fdimage0.bin
	../z_tools/make.exe -C ../z_tools/qemu

install :
	../z_tools/make.exe img
	../z_tools/imgtol.com w a: helloos.img

clean :
	-del ipl.bin
	-del ipl.lst

src_only :
	../z_tools/make.exe clean
	-del helloos.img
\end{code}

使用方法为:用!cons 打开命令行窗口,然后就可以通过输入 |make img|来生成映像文件,输入 |make run|来运行,等等。可以使用的参数在 Makefile 文件中写明了。另外,当执行不带参数的make命令时,相当于执行 |make img|。







