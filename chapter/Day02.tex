\chapter{	汇编语言学习与Makefile入门	}
\section{	介绍文本编辑器	}
如中文译者所注,推荐使用Notepad++文本编辑器。

使用这个软件打开光盘中提供的源代码会出现日语注释乱码,选择~格式->编码字符集-> 日文->ShiftJIS~即可正常显示代码中原书作者的日语注释。

原书作者推荐的文本编辑器TeraPad在中文系统中会出现软件本身的文本乱码,如菜单栏工具栏的文字,但是无需设置就可以正常显示代码日语注释。

\section{	继续开发	}

\begin{code}[label=helloos.nas 节选]
; hello-os
; TAB=4

		ORG		0x7c00			; このプログラムがどこに読み込まれるのか

; 以下は標準的なFAT12フォーマットフロッピーディスクのための記述

		JMP		entry
		DB		0x90
---中略---
; プログラム本体

entry:
		MOV		AX,0			; レジスタ初期化
		MOV		SS,AX
		MOV		SP,0x7c00
		MOV		DS,AX
		MOV		ES,AX

		MOV		SI,msg
putloop:
		MOV		AL,[SI]
		ADD		SI,1			; SIに1を足す
		CMP		AL,0
		JE		fin
		MOV		AH,0x0e			; 一文字表示ファンクション
		MOV		BX,15			; カラーコード
		INT		0x10			; ビデオBIOS呼び出し
		JMP		putloop
fin:
		HLT						; 何かあるまでCPUを停止させる
		JMP		fin				; 無限ループ

msg:
		DB		0x0a, 0x0a		; 改行を2つ
		DB		"hello, world"
		DB		0x0a			; 改行
		DB		0

		RESB	0x7dfe-$		; 0x7dfeまでを0x00で埋める命令

		DB		0x55, 0xaa

; 以下はブートセクタ以外の部分の記述

		DB		0xf0, 0xff, 0xff, 0x00, 0x00, 0x00, 0x00, 0x00
		RESB	4600
		DB		0xf0, 0xff, 0xff, 0x00, 0x00, 0x00, 0x00, 0x00
		RESB	1469432  
\end{code}
\section{	先制作启动区	}
\section{	Makefile入门	}

