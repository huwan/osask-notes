\chapter{	C语言与画面显示的练习	}
\section{	用C语言实现内存写入(harib01a)	}
在让画面黑屏的基础上,通过写VRAM的值,在画面上画出些“花”来。

\dag|projects\04_day\harib01a|
\begin{code}[label=naskfunc.nas]
_write_mem8:	; void write_mem8(int addr, int data);
		MOV		ECX,[ESP+4]		; [ESP+4]にaddrが入っているのでそれをECXに読み込む
		MOV		AL,[ESP+8]		; [ESP+8]にdataが入っているのでそれをALに読み込む
		MOV		[ECX],AL
		RET
\end{code}

C语言部分:
\begin{code}[label=bootpack.c]
void io_hlt(void);
void write_mem8(int addr, int data);

void HariMain(void)
{
	int i; /* 変数宣言。iという変数は、32ビットの整数型 */

	for (i = 0xa0000; i <= 0xaffff; i++) {
		write_mem8(i, 15); /* MOV BYTE [i],15 */
	}

	for (;;) {
		io_hlt();
	}
}
\end{code}

VRAM中都写入了15,意思是全部像素的颜色都是第15种颜色,即白色,因此运行程序后画面会变成白色。
\section{	条纹图案(harib01b)	}
\dag|projects\04_day\harib01b|
\begin{code}[label=bootpack.c]

	for (i = 0xa0000; i <= 0xaffff; i++) {
		write_mem8(i, i&0x0f); 
	}
\end{code}

通过与运算,将15改成特殊的值,低4位保持不变,高4位全部变成0。这样每隔16个像素,色号就反复一次。


\section{	挑战指针(harib01c)	}
使用C语言的指针,修改上面程序,实现内存写入。

\dag|projects\04_day\harib01c|
\begin{code}
void io_hlt(void);

void HariMain(void)
{
	int i; /* 変数宣言。iという変数は、32ビットの整数型 */
	char *p; /* pという変数は、BYTE [...]用の番地 */

	for (i = 0xa0000; i <= 0xaffff; i++) {

		p = i; /* 番地を代入 */
		*p = i & 0x0f;

		/* これで write_mem8(i, i & 0x0f); の代わりになる */
	}

	for (;;) {
		io_hlt();
	}
}
\end{code}
\section{	指针的应用(1)(harib01d)	}
\section{	指针的应用(2)(harib01e)	}
\section{	色号设定(harib01f)	}
\section{	绘制矩形(harib01g)	}
\section{	今天的成果(harib01h)	}

