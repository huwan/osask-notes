\chapter{	内存管理	}
\section{	整理源文件(harib06a)	}
鼠标指针的叠加处理留待以后处理,先转向内存管理。

把bootpack.c里面的一些函数单独择出去。
\begin{table}[!ht]
  \centering
  \begin{tabular}{|l|l|l|}
  \hline
  % after \\: \hline or \cline{col1-col2} \cline{col3-col4} ...
  函数名 & 移动前 & 移动后 \\\hline
    wait\_KBC\_sendready&	bootpack.c&	keyboard.c\\
    init\_keyboard&	bootpack.c&	keyboard.c\\
    enable\_mouse&	bootpack.c&	mouse.c\\
    mouse\_decode&	bootpack.c&	mouse.c\\
    inthandler21&	init.c&	keyboard.c\\
    inthandler2c&	init.c&	mouse.c\\
    \hline
\end{tabular}
\end{table}

\section{	内存容量检查(1)(harib06b)	}

在进行内存管理之前,必须要做的事情是搞清楚内存究竟到底有多大,范围是到哪里。

在最初启动时,BIOS肯定要检查内存容量,所以只要我们问一问BIOS,就能知道内存容量有多大。但问题是,如果那样做的话,一方面asmhead.nas会变长,另一方面,BIOS版本不同,BIOS函数的调用方法也不相同,麻烦事太多了。所以,作者自己去检查内存。

内存检查时,要往内存里随便写入一个值,然后马上读取,来检查读取的值与写入的值是否相等。如果内存连接正常,则写入的值能够记在内存里。如果没连接上,则读出的值肯定是乱七八糟的。但是,如果CPU里加上了缓存会导致写入和读出的不是内存,而是缓存。结果,所有的内存都“正常”,检查处理不能完成。

所以,只有在内存检查时才将缓存设为OFF。具体来说,就是先查查CPU是不是在486以上,如果是,就将缓存设为OFF。按照这一思路,我们创建了以下函数memtest。

\begin{code}[label=bootpack.c节选]
#define EFLAGS_AC_BIT		0x00040000
#define CR0_CACHE_DISABLE	0x60000000

unsigned int memtest(unsigned int start, unsigned int end)
{
	char flg486 = 0;
	unsigned int eflg, cr0, i;

	/* 386か、486以降なのかの確認 */
	eflg = io_load_eflags();
	eflg |= EFLAGS_AC_BIT; /* AC-bit = 1 */
	io_store_eflags(eflg);
	eflg = io_load_eflags();
	if ((eflg & EFLAGS_AC_BIT) != 0) { /* 386ではAC=1にしても自動で0 に戻ってしまう */
		flg486 = 1;
	}
	eflg &= ~EFLAGS_AC_BIT; /* AC-bit = 0 */
	io_store_eflags(eflg);

	if (flg486 != 0) {
		cr0 = load_cr0();
		cr0 |= CR0_CACHE_DISABLE; /* キャッシュ禁止 */
		store_cr0(cr0);
	}

	i = memtest_sub(start, end);

	if (flg486 != 0) {
		cr0 = load_cr0();
		cr0 &= ~CR0_CACHE_DISABLE; /* キャッシュ許可 */
		store_cr0(cr0);
	}

	return i;
}
\end{code}

最初对EFLAGS进行的处理,是检查CPU是486以上还是386。如果是486以上,EFLAGS寄存器的第18位应该是所谓的AC标志位;如果CPU是386,那么就没有这个标志位,第18位一直是0。这里,我们有意识地把1写入到这一位,然后再读出EFLAGS的值,继而检查AC标志位是否仍为1。最后,将AC标志位重置为0。

将AC标志位重置为0时,用到了AND运算,那里出现了一个运算符“~”,它是取反运算符,就是将所有的位都反转的意思。所以,|~EFLAGS_AC_BIT|与0xfffbffff 一样。

为了禁止缓存,需要对CR0寄存器的某一标志位进行操作,需要用到函数|load_cr0| 和|store_cr0|,这个函数存在naskfunc.nas里。

\begin{code}[label=naskfunc.nas节选]
_load_cr0:		; int load_cr0(void);
		MOV		EAX,CR0
		RET

_store_cr0:		; void store_cr0(int cr0);
		MOV		EAX,[ESP+4]
		MOV		CR0,EAX
		RET
\end{code}

\cs

|memtest_sub|函数,是内存检查处理的实现部分。

\begin{code}
unsigned int memtest_sub(unsigned int start, unsigned int end)
{
	unsigned int i, *p, old, pat0 = 0xaa55aa55, pat1 = 0x55aa55aa;
	for (i = start; i <= end; i += 0x1000) {
		p = (unsigned int *) (i + 0xffc);
		old = *p;			/* いじる前の値を覚えておく */
		*p = pat0;			/* ためしに書いてみる */
		*p ^= 0xffffffff;	/* そしてそれを反転してみる */
		if (*p != pat1) {	/* 反転結果になったか? */
not_memory:
			*p = old;
			break;
		}
		*p ^= 0xffffffff;	/* もう一度反転してみる */
		if (*p != pat0) {	/* 元に戻ったか? */
			goto not_memory;
		}
		*p = old;			/* いじった値を元に戻す */
	}
	return i;
}
\end{code}

调查从start地址到end地址的范围内,能够使用的内存的末尾地址。首先如果p 不是指针,就不能指定地址去读取内存,所以先执行“p=i;”。紧接着使用这个p,将原值保存下来(变量old)。接着试写0xaa55aa55,在内存里反转该值,检查结果是否正确。如果正确,就再次反转它,检查一下是否能回复到初始值。最后,使用old变量,将内存的值恢复回去。如果在某个环节没能恢复成预想的值,那么就在那个环节终止调查,并报告终止时的地址。

for 语句中i 的增值部分以及p的赋值部分。每次只增加4,就要检查全部内存,速度太慢了,所以改成了每次增加0x1000,相当于4KB,这样一来速度就提高了1000倍。p的赋值计算式也变了,这是因为,如果不进行任何改变仍写作“p=i;”的话,程序就会只检查4KB最开头的4个字节。所以要改为“p=i + 0xffc;”,让它只检查末尾的4个字节。


\cs

修改HariMain程序,添加以下部分:
\begin{code}[label=bootpack.c节选]
	i = memtest(0x00400000, 0xbfffffff) / (1024 * 1024);
	sprintf(s, "memory %dMB", i);
	putfonts8_asc(binfo->vram, binfo->scrnx, 0, 32, COL8_FFFFFF, s);
\end{code}

暂时先使用以上程序对0x00400000~0xbfffffff范围的内存进行检查。这个程序最大可以识别3GB范围的内存。0x00400000号以前的内存已经被使用了(参考8.5节的内存分布图),没有内存,程序根本运行不到这里,所以我们没做内存检查。以MB为单位。

如果在QEMU上运行,根据模拟器的设定,内存应该为32MB。不过运行结果显示的是3072MB。有错误。

\section{	内存容量检查(2)(harib06c)	}

真正的原因是编译器对程序进行了优化,导致写的程序中有一些直接被「优化」掉了。

使用“make -r bootpack.nas”来运行的话,就可以确认bootpack.c被编译成了什么样的机器语言。用文本编辑器看一看生成的bootpack.nas会发现,最下边有|memtest_sub|的编译结果。

\begin{code}[label=memtest\_sub的编译结果]
_memtest_sub:
    PUSH    EBP                 ; C编译器的固定语句
    MOV EBP,ESP
    MOV EDX,DWORD [12+EBP]      ; EDX = end;
    MOV EAX,DWORD [8+EBP]       ; EAX = start; /* EAX是i */
    CMP EAX,EDX                 ; if (EAX > EDX) goto L30;
    JA  L30
L36:
L34:
    ADD EAX,4096                ; EAX += 0x1000;
    CMP EAX,EDX                 ; if (EAX <= EDX) goto L36;
    JBE L36
L30:
    POP EBP                     ; 接收前文中PUSH的EBP
    RET                         ; return;
\end{code}

与原始程序相比会发现,编译后没有XOR等指令,而且,好像编译后只剩下了for 语句。

\cs

编译器优化过程:

首先将内存的内容保存到old里,然后写入pat0的值,再反转,最后跟pat1进行比较。肯定相等,if语句不成立,得不到执行,所以把它删掉,将比较*p和pat0这部分也删掉。
\begin{code}
 unsigned int memtest_sub(unsigned int start, unsigned int end)
  {
      unsigned int i, *p, old, pat0 = 0xaa55aa55, pat1 = 0x55aa55aa;
      for (i = start; i <= end; i += 0x1000) {
          p = (unsigned int *) (i + 0xffc);

          old = *p;           /* 先记住修改前的值*/
          *p = pat0;          /* 试写 */
          *p ^= 0xffffffff;   /* 反转 */
          *p ^= 0xffffffff;   /* 再次反转 */
          *p = old;           /* 恢复为修改前的值 */
      }
      return i;
  }
\end{code}

反转了两次会变回之前的状态,所以这些处理也可以不要。
\begin{code}
unsigned int memtest_sub(unsigned int start, unsigned int end)
  {
      unsigned int i, *p, old, pat0 = 0xaa55aa55, pat1 = 0x55aa55aa;
      for (i = start; i <= end; i += 0x1000) {
          p = (unsigned int *) (i + 0xffc);
          old = *p;           /* 先记住修改前的值 */
          *p = pat0;          /* 试写 */
          *p = old;           /* 恢复为修改前的值 */
      }
      return i;
  }
\end{code}

还有,“*p = pat0;”本来就没有意义。反正要将old的值赋给*p。
\begin{code}
unsigned int memtest_sub(unsigned int start, unsigned int end)
  {
      unsigned int i, *p, old, pat0 = 0xaa55aa55, pat1 = 0x55aa55aa;
      for (i = start; i <= end; i += 0x1000) {
          p = (unsigned int *) (i + 0xffc);
          old = *p;           /* 先记住修改前的值 */
          *p = old;           /* 恢复为修改前的值 */
      }
      return i;
  }
\end{code}

*p里面实际没写进任何内容,删。
\begin{code}
unsigned int memtest_sub(unsigned int start, unsigned int end)
  {
      unsigned int i, *p, old, pat0 = 0xaa55aa55, pat1 = 0x55aa55aa;
      for (i = start; i <= end; i += 0x1000) {
          p = (unsigned int *) (i + 0xffc);
      }
      return i;
  }
\end{code}

这里的地址变量p,虽然计算了地址,却一次也没有用到,old、pat0、pat1 也都是用不到的变量。全部都舍弃掉。

\begin{code}
 unsigned int memtest_sub(unsigned int start, unsigned int end)
  {
      unsigned int i;
      for (i = start; i <= end; i += 0x1000) { }
      return i;
  }
\end{code}

根据以上编译器的思路,我们可以看出,它进行了最优化处理。但其实这个工作本来是不需要的。用于应用程序的C编译器,根本想不到会对没有内存的地方进行读写。

如果更改编译选项,是可以停止最优化处理的。可是在其他地方,我们还是需要如此考虑周密的最优化处理的,所以不想更改编译选项。于是决定|memtest_sub|也用汇编来写。

\begin{code}[label=naskfunc.nas节选]
_memtest_sub:   ; unsigned int memtest_sub(unsigned int start, unsigned int end)
		PUSH	EDI						; (EBX, ESI, EDI も使いたいので)
        PUSH    ESI
        PUSH    EBX
        MOV     ESI,0xaa55aa55          ; pat0 = 0xaa55aa55;
        MOV     EDI,0x55aa55aa          ; pat1 = 0x55aa55aa;
        MOV     EAX,[ESP+12+4]          ; i = start;
mts_loop:
        MOV     EBX,EAX
        ADD     EBX,0xffc               ; p = i + 0xffc;
        MOV     EDX,[EBX]               ; old = *p;
        MOV     [EBX],ESI               ; *p = pat0;
        XOR     DWORD [EBX],0xffffffff  ; *p ^= 0xffffffff;
        CMP     EDI,[EBX]               ; if (*p != pat1) goto fin;
        JNE     mts_fin
        XOR     DWORD [EBX],0xffffffff  ; *p ^= 0xffffffff;
        CMP     ESI,[EBX]               ; if (*p != pat0) goto fin;
        JNE     mts_fin
        MOV     [EBX],EDX               ; *p = old;
        ADD     EAX,0x1000              ; i += 0x1000;
        CMP     EAX,[ESP+12+8]          ; if (i <= end) goto mts_loop;

       JBE     mts_loop
        POP     EBX
        POP     ESI
        POP     EDI
        RET
mts_fin:
        MOV     [EBX],EDX               ; *p = old;
        POP     EBX
        POP     ESI
        POP     EDI
        RET
\end{code}

删除bootpack.c中的|memtest_sub|函数,运行程序。内存容量显示正常。
\section{	挑战内存管理(harib06d)	}
内存管理的基础,一是内存分配,一是内存释放。\footnote{建议找本操作系统的书看看各种内存管理方法。这里不想写了。}

内存管理程序:
\begin{code}[label=bootpack.c节选]
#define MEMMAN_FREES		4090	/* 偙傟偱栺32KB */

struct FREEINFO {	/* あき情報 */
	unsigned int addr, size;
};

struct MEMMAN {		/* メモリ管理 */
	int frees, maxfrees, lostsize, losts;
	struct FREEINFO free[MEMMAN_FREES];
};

void memman_init(struct MEMMAN *man)
{
	man->frees = 0;			/* あき情報の個数 */
	man->maxfrees = 0;		/* 状況観察用:freesの最大値 */
	man->lostsize = 0;		/* 解放に失敗した合計サイズ */
	man->losts = 0;			/* 解放に失敗した回数 */
	return;
}

unsigned int memman_total(struct MEMMAN *man)
/* あきサイズの合計を報告 */
{
	unsigned int i, t = 0;
	for (i = 0; i < man->frees; i++) {
		t += man->free[i].size;
	}
	return t;
}

unsigned int memman_alloc(struct MEMMAN *man, unsigned int size)
/* 確保 */
{
	unsigned int i, a;
	for (i = 0; i < man->frees; i++) {
		if (man->free[i].size >= size) {
			/* 十分な広さのあきを発見 */
			a = man->free[i].addr;
			man->free[i].addr += size;
			man->free[i].size -= size;
			if (man->free[i].size == 0) {
				/* free[i]がなくなったので前へつめる */
				man->frees--;
				for (; i < man->frees; i++) {
					man->free[i] = man->free[i + 1]; /* 構造体の代入 */
				}
			}
			return a;
		}
	}
	return 0; /* あきがない */
}
\end{code}

一开始的struct MEMMAN,创建了4000 组,留出不少余量,管理空间大约是32KB。其中还有变量 maxfrees、lostsize、losts等,这些变量与管理本身没有关系,不用在意它们。如果特别想了解的话,可以看看函数|memman_init|的注释,里面有介绍。

函数|memman_init|对memman进行了初始化,设定为空。主要工作,是将frees 设为0,而其他的都是附属性设定。

函数|memman_total|用来计算可用内存的合计大小并返回。

最后的|memman_alloc|函数,功能是分配指定大小的内存。除了|free[i].size| 变为0时的处理以外的部分,在前面已经说过了。

|memman_alloc|函数中|free[i].size|等于0的处理,与FIFO缓冲区的处理方法很相似,要进行移位处理。希望大家注意以下写法:
\begin{code}
man->free[i].addr = man->free[i+1].addr;

man->free[i].size = man->free[i+1].size;
\end{code}


我们在这里将其归纳为了:
\begin{code}
man->free[i] = man->free[i+1];
\end{code}

这种方法被称为结构体赋值,其使用方法如上所示,可以写成简单的形式。

\cs

释放内存函数,也就是往memman里追加可用内存信息的函数,稍微有点复杂。

\begin{code}[label=bootpack.c节选]
int memman_free(struct MEMMAN *man, unsigned int addr, unsigned int size)
/* 解放 */
{
	int i, j;
	/* まとめやすさを考えると、free[]がaddr順に並んでいるほうがいい */
	/* だからまず、どこに入れるべきかを決める */
	for (i = 0; i < man->frees; i++) {
		if (man->free[i].addr > addr) {
			break;
		}
	}
	/* free[i - 1].addr < addr < free[i].addr */
	if (i > 0) {
		/* 前がある */
		if (man->free[i - 1].addr + man->free[i - 1].size == addr) {
			/* 前のあき領域にまとめられる */
			man->free[i - 1].size += size;
			if (i < man->frees) {
				/* 後ろもある */
				if (addr + size == man->free[i].addr) {
					/* なんと後ろともまとめられる */
					man->free[i - 1].size += man->free[i].size;
					/* man->free[i]の削除 */
					/* free[i]がなくなったので前へつめる */
					man->frees--;
					for (; i < man->frees; i++) {
						man->free[i] = man->free[i + 1]; /* 構造体の代入 */
					}
				}
			}
			return 0; /* 成功終了 */
		}
	}
	/* 前とはまとめられなかった */
	if (i < man->frees) {
		/* 後ろがある */
		if (addr + size == man->free[i].addr) {
			/* 後ろとはまとめられる */
			man->free[i].addr = addr;
			man->free[i].size += size;
			return 0; /* 成功終了 */
		}
	}
	/* 前にも後ろにもまとめられない */
	if (man->frees < MEMMAN_FREES) {
		/* free[i]より後ろを、後ろへずらして、すきまを作る */
		for (j = man->frees; j > i; j--) {
			man->free[j] = man->free[j - 1];
		}
		man->frees++;
		if (man->maxfrees < man->frees) {
			man->maxfrees = man->frees; /* 最大値を更新 */
		}
		man->free[i].addr = addr;
		man->free[i].size = size;
		return 0; /* 成功終了 */
	}
	/* 後ろにずらせなかった */
	man->losts++;
	man->lostsize += size;
	return -1; /* 失敗終了 */
}
\end{code}

如果可用信息表满了,就按照舍去之后带来损失最小的原则进行割舍。但是在这个程序里,我们并没有对损失程度进行比较,而是舍去了刚刚进来的可用信息,这只是为了图个方便。

\cs

最后,将这个程序应用于HariMain,结果就变成了下面这样。

\begin{code}[label=bootpack.c节选]
#define MEMMAN_ADDR         0x003c0000

void HariMain(void)
{
    (中略)
    unsigned int memtotal;
    struct MEMMAN *memman = (struct MEMMAN *) MEMMAN_ADDR;
    (中略)
    memtotal = memtest(0x00400000, 0xbfffffff);
    memman_init(memman);
    memman_free(memman, 0x00001000, 0x0009e000); /* 0x00001000 - 0x0009efff */
    memman_free(memman, 0x00400000, memtotal - 0x00400000);
    (中略)
    sprintf(s, "memory %dMB   free : %dKB",
            memtotal / (1024 * 1024), memman_total(memman) / 1024);
    putfonts8_asc(binfo->vram, binfo->scrnx, 0, 32, COL8_FFFFFF, s);
\end{code}

memman需要32KB,我们暂时决定使用自0x003c0000开始的32KB(0x00300000号地址以后,今后的程序即使有所增加,预计也不会到达0x003c0000,所以我们使用这一数值),然后计算内存总量memtotal,将现在不用的内存以0x1000个字节为单位注册到memman里。最后,显示出合计可用内存容量。在QEMU上执行时,有时会注册成632KB和28MB。$632+28672=29304$,所以屏幕上会显示出29304KB。

 