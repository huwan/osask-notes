\chapter{	鼠标控制与32位模式切换	}
\section{	鼠标解读(1)(harib05a)	}
在上一章得到了鼠标的数据,现在要解读这些数据,然后结合鼠标的动作,让鼠标指针动起来。

\begin{code}
	unsigned char mouse_dbuf[3], mouse_phase;
	enable_mouse();
	mouse_phase = 0; /* マウスの0xfaを待っている段階へ */

	for (;;) {
		io_cli();
		if (fifo8_status(&keyfifo) + fifo8_status(&mousefifo) == 0) {
			io_stihlt();
		} else {
			if (fifo8_status(&keyfifo) != 0) {
				i = fifo8_get(&keyfifo);
				io_sti();
				sprintf(s, "%02X", i);
				boxfill8(binfo->vram, binfo->scrnx, COL8_008484,  0, 16, 15, 31);
				putfonts8_asc(binfo->vram, binfo->scrnx, 0, 16, COL8_FFFFFF, s);
			} else if (fifo8_status(&mousefifo) != 0) {
				i = fifo8_get(&mousefifo);
				io_sti();
				if (mouse_phase == 0) {
					/* マウスの0xfaを待っている段階 */
					if (i == 0xfa) {
						mouse_phase = 1;
					}
				} else if (mouse_phase == 1) {
					/* マウスの1バイト目を待っている段階 */
					mouse_dbuf[0] = i;
					mouse_phase = 2;
				} else if (mouse_phase == 2) {
					/* マウスの2バイト目を待っている段階 */
					mouse_dbuf[1] = i;
					mouse_phase = 3;
				} else if (mouse_phase == 3) {
					/* マウスの3バイト目を待っている段階 */
					mouse_dbuf[2] = i;
					mouse_phase = 1;
					/* データが3バイト揃ったので表示 */
					sprintf(s, "%02X %02X %02X", mouse_dbuf[0], mouse_dbuf[1], mouse_dbuf[2]);
					boxfill8(binfo->vram, binfo->scrnx, COL8_008484, 32, 16, 32 + 8 * 8 - 1, 31);
					putfonts8_asc(binfo->vram, binfo->scrnx, 32, 16, COL8_FFFFFF, s);
				}
			}
		}
	}
\end{code}

首先要把最初读到的0xfa舍弃掉。之后,每次从鼠标那里送过来的数据都应该是3个字节一组的,所以每当数据累积到3个字节,就把它显示在屏幕上。

变量|mouse_phase|用来记住接收鼠标数据的工作进展到了什么阶段(phase)。接收到的数据放在|mouse_dbuf[0~2]|内。

\cs

运行程序,点击鼠标或者滚动鼠标,可以看到各种反应。

鼠标的3个字节的数据分别表示点击(左击,右击,滚轮)数据,鼠标左右移动,鼠标上下移动。

\section{	稍事整理(harib05b)	}
\begin{code}[label=修改后的bootpack.c节选]
/* bootpackのメイン */

#include "bootpack.h"
#include <stdio.h>

struct MOUSE_DEC {
	unsigned char buf[3], phase;
};

void enable_mouse(struct MOUSE_DEC *mdec);
int mouse_decode(struct MOUSE_DEC *mdec, unsigned char dat);

void HariMain(void)
{
    (中略)
	struct MOUSE_DEC mdec;
     (中略)

	enable_mouse(&mdec);

	for (;;) {
		io_cli();
		if (fifo8_status(&keyfifo) + fifo8_status(&mousefifo) == 0) {
			io_stihlt();
		} else {
			if (fifo8_status(&keyfifo) != 0) {
				i = fifo8_get(&keyfifo);
				io_sti();
				sprintf(s, "%02X", i);
				boxfill8(binfo->vram, binfo->scrnx, COL8_008484,  0, 16, 15, 31);
				putfonts8_asc(binfo->vram, binfo->scrnx, 0, 16, COL8_FFFFFF, s);
			} else if (fifo8_status(&mousefifo) != 0) {
				i = fifo8_get(&mousefifo);
				io_sti();
				if (mouse_decode(&mdec, i) != 0) {
					/* データが3バイト揃ったので表示 */
					sprintf(s, "%02X %02X %02X", mdec.buf[0], mdec.buf[1], mdec.buf[2]);
					boxfill8(binfo->vram, binfo->scrnx, COL8_008484, 32, 16, 32 + 8 * 8 - 1, 31);
					putfonts8_asc(binfo->vram, binfo->scrnx, 32, 16, COL8_FFFFFF, s);
				}
			}
		}
	}
}

void enable_mouse(struct MOUSE_DEC *mdec)
{
	/* マウス有効 */
	wait_KBC_sendready();
	io_out8(PORT_KEYCMD, KEYCMD_SENDTO_MOUSE);
	wait_KBC_sendready();
	io_out8(PORT_KEYDAT, MOUSECMD_ENABLE);
	/* うまくいくとACK(0xfa)が送信されてくる */
	mdec->phase = 0; /* マウスの0xfaを待っている段階 */
	return;
}

int mouse_decode(struct MOUSE_DEC *mdec, unsigned char dat)
{
	if (mdec->phase == 0) {
		/* マウスの0xfaを待っている段階 */
		if (dat == 0xfa) {
			mdec->phase = 1;
		}
		return 0;
	}
	if (mdec->phase == 1) {
		/* マウスの1バイト目を待っている段階 */
		mdec->buf[0] = dat;
		mdec->phase = 2;
		return 0;
	}
	if (mdec->phase == 2) {
		/* マウスの2バイト目を待っている段階 */
		mdec->buf[1] = dat;
		mdec->phase = 3;
		return 0;
	}
	if (mdec->phase == 3) {
		/* マウスの3バイト目を待っている段階 */
		mdec->buf[2] = dat;
		mdec->phase = 1;
		return 1;
	}
	return -1; /* ここに来ることはないはず */
}

\end{code}

整理了程序。创建了一个结构体|MOUSE_DEC|,把解读鼠标所需要的变量都归总到一块儿。

在函数|enable_mouse|的最后,添加了将phase归零的处理。之所以要舍去读到的0xfa,是因为鼠标已经激活了。因此我们进行归零处理也不错。

将鼠标的解读从函数HariMain的接收信息处理中剥离出来,放到了|mouse_decode|函数里, Harimain又回到了清晰的状态。3个字节凑齐后,|mouse_decode|函数执行“return 1;”,把这些数据显示出来。



\section{	鼠标解读(2)(harib05c)	}
对|mouse_decode|函数略加修改。
\begin{code}
struct MOUSE_DEC {
	unsigned char buf[3], phase;
	int x, y, btn;
};

int mouse_decode(struct MOUSE_DEC *mdec, unsigned char dat)
{
	if (mdec->phase == 0) {
		/* マウスの0xfaを待っている段階 */
		if (dat == 0xfa) {
			mdec->phase = 1;
		}
		return 0;
	}
	if (mdec->phase == 1) {
		/* マウスの1バイト目を待っている段階 */
		if ((dat & 0xc8) == 0x08) {
			/* 正しい1バイト目だった */
			mdec->buf[0] = dat;
			mdec->phase = 2;
		}
		return 0;
	}
	if (mdec->phase == 2) {
		/* マウスの2バイト目を待っている段階 */
		mdec->buf[1] = dat;
		mdec->phase = 3;
		return 0;
	}
	if (mdec->phase == 3) {
		/* マウスの3バイト目を待っている段階 */
		mdec->buf[2] = dat;
		mdec->phase = 1;
		mdec->btn = mdec->buf[0] & 0x07;
		mdec->x = mdec->buf[1];
		mdec->y = mdec->buf[2];
		if ((mdec->buf[0] & 0x10) != 0) {
			mdec->x |= 0xffffff00;
		}
		if ((mdec->buf[0] & 0x20) != 0) {
			mdec->y |= 0xffffff00;
		}
		mdec->y = - mdec->y; /* マウスではy方向の符号が画面と反対 */
		return 1;
	}
	return -1; /* ここに来ることはないはず */
}
\end{code}

结构体里增加的几个变量用于存放解读结果。这几个变量是x、y和btn,分别用于存放移动信息和鼠标按键状态。

还修改了if(mdec—>phase==1)语句。这个if语句,用于判断第一字节对移动有反应的部分是否在0~3的范围内;同时还要判断第一字节对点击有反应的部分是否在8~F的范围内。如果这个字节的数据不在以上范围内,它就会被舍去。

最后的if(mdec—>phase==3)部分,是解读处理的核心。鼠标键的状态,放在|buf[0]|的低3位,我们只取出这3位。十六进制的0x07相当于二进制的0000 0111,因此通过与运算(|&|),可以很顺利地取出低3位的值。

x和y,基本上是直接使用|buf[1]|和|buf[2]| ,但是需要使用第一字节中对鼠标移动有反应的几位(参考第一节的叙述)信息,将x和y的第8 位及第8 位以后全部都设成1,或全部都保留为0。这样就能正确地解读x和y。

在解读处理的最后,对y的符号进行了取反的操作。这是因为,鼠标与屏幕的y 方向正好相反,为了配合画面方向,就对y符号进行了取反操作。


\section{	移动鼠标指针(harib05d)	}
修改图形显示部分,让鼠标在屏幕上动起来。
\begin{code}[label=HariMain节选]
			} else if (fifo8_status(&mousefifo) != 0) {
				i = fifo8_get(&mousefifo);
				io_sti();
				if (mouse_decode(&mdec, i) != 0) {
					/* データが3バイト揃ったので表示 */
					sprintf(s, "[lcr %4d %4d]", mdec.x, mdec.y);
					if ((mdec.btn & 0x01) != 0) {
						s[1] = 'L';
					}
					if ((mdec.btn & 0x02) != 0) {
						s[3] = 'R';
					}
					if ((mdec.btn & 0x04) != 0) {
						s[2] = 'C';
					}
					boxfill8(binfo->vram, binfo->scrnx, COL8_008484, 32, 16, 32 + 15 * 8 - 1, 31);
					putfonts8_asc(binfo->vram, binfo->scrnx, 32, 16, COL8_FFFFFF, s);
					/* マウスカーソルの移動 */
					boxfill8(binfo->vram, binfo->scrnx, COL8_008484, mx, my, mx + 15, my + 15); /* マウス消す */
					mx += mdec.x;
					my += mdec.y;
					if (mx < 0) {
						mx = 0;
					}
					if (my < 0) {
						my = 0;
					}
					if (mx > binfo->scrnx - 16) {
						mx = binfo->scrnx - 16;
					}
					if (my > binfo->scrny - 16) {
						my = binfo->scrny - 16;
					}
					sprintf(s, "(%3d, %3d)", mx, my);
					boxfill8(binfo->vram, binfo->scrnx, COL8_008484, 0, 0, 79, 15); /* 座標消す */
					putfonts8_asc(binfo->vram, binfo->scrnx, 0, 0, COL8_FFFFFF, s); /* 座標書く */
					putblock8_8(binfo->vram, binfo->scrnx, 16, 16, mx, my, mcursor, 16); /* マウス描く */
				}
			}
\end{code}

程序中会检查mdec.btn的值,用3个if语句将s的值置换成相应的字符串。

隐藏掉鼠标指针,然后在鼠标指针的坐标上,加上解读得到的位移量。“mx += mdec.x;”是“mx = mx + mdec.x;”的省略形式。因为不能让鼠标指针跑到屏幕外面去,所以进行了调整,调整后重新显示鼠标坐标,鼠标指针也会重新描画。

此时鼠标已经可以正常动起来了。不过只要鼠标一接触到装饰在屏幕下部的任务栏,就会有擦除效果。这是因为我们没有考虑到叠加处理,所以画面就出问题了。

\section{	通往32位模式之路	}
本节说明asmhead.nas中的如同谜一样的大约100行程序。

在没有说明的这段程序中,最开始做的事情如下:

\begin{code}[label=asmhead.nas节选]
; PICが一切の割り込みを受け付けないようにする
;	AT互換機の仕様では、PICの初期化をするなら、
;	こいつをCLI前にやっておかないと、たまにハングアップする
;	PICの初期化はあとでやる

		MOV		AL,0xff
		OUT		0x21,AL
		NOP						; OUT命令を連続させるとうまくいかない機種があるらしいので
		OUT		0xa1,AL

		CLI						; さらにCPUレベルでも割り込み禁止
\end{code}

这段程序等同于以下内容的C程序。
\begin{code}
io_out(PIC0_IMR, 0xff); /* 禁止主PIC的全部中断 */
io_out(PIC1_IMR, 0xff); /* 禁止从PIC的全部中断 */
io_cli(); /* 禁止CPU级别的中断*/
\end{code}

如果当CPU进行模式转换时进来了中断信号,那可就麻烦了。而且,后来还要进行PIC的初始化,初始化时也不允许有中断发生。所以,要把中断全部屏蔽掉。

\cs

\begin{code}[label=asmhead.nas节选(续)]
; CPUから1MB以上のメモリにアクセスできるように、A20GATEを設定

		CALL	waitkbdout
		MOV		AL,0xd1
		OUT		0x64,AL
		CALL	waitkbdout
		MOV		AL,0xdf			; enable A20
		OUT		0x60,AL
		CALL	waitkbdout
\end{code}

这里的waitkbdout,等同于|wait_KBC_sendready|。这段程序在C语言里的写法大致如下:
\begin{code}
#define KEYCMD_WRITE_OUTPORT    0xd1
#define KBC_OUTPORT_A20G_ENABLE 0xdf

    /* A20GATE的设定 */
    wait_KBC_sendready();
    io_out8(PORT_KEYCMD, KEYCMD_WRITE_OUTPORT);
    wait_KBC_sendready();
    io_out8(PORT_KEYDAT, KBC_OUTPORT_A20G_ENABLE);
    wait_KBC_sendready(); /* 这句话是为了等待完成执行指令 */
\end{code}


程序的基本结构与|init_keyboard|完全相同,功能仅仅是往键盘控制电路发送指令。

这里发送的指令,是指令键盘控制电路的附属端口输出0xdf。这个附属端口,连接着主板上的很多地方,通过这个端口发送不同的指令,就可以实现各种各样的控制功能。

这次输出0xdf所要完成的功能,是让A20GATE信号线变成ON的状态。它能使内存的1MB以上的部分变成可使用状态。最初出现电脑的时候,CPU只有16位模式,所以内存最大也只有1MB。后来CPU变聪明了,可以使用很大的内存了。但为了兼容旧版的操作系统,在执行激活指令之前,电路被限制为只能使用1MB内存。A20GATE信号线正是用来使这个电路停止从而让所有内存都可以使用的东西。

最后还有一点,“|wait_KBC_sendready();|”是多余的。在此之后,虽然不会往键盘送命令,但仍然要等到下一个命令能够送来为止。这是为了等待A20GATE的处理切实完成。

\cs

\begin{code}[label=asmhead.nas节选 (续)]
; プロテクトモード移行

[INSTRSET "i486p"]				; 486の命令まで使いたいという記述

		LGDT	[GDTR0]			; 暫定GDTを設定
		MOV		EAX,CR0
		AND		EAX,0x7fffffff	; bit31を0にする(ページング禁止のため)
		OR		EAX,0x00000001	; bit0を1にする(プロテクトモード移行のため)
		MOV		CR0,EAX
		JMP		pipelineflush
pipelineflush:
		MOV		AX,1*8			;  読み書き可能セグメント32bit
		MOV		DS,AX
		MOV		ES,AX
		MOV		FS,AX
		MOV		GS,AX
		MOV		SS,AX
\end{code}

INSTRSET指令,是为了能够使用386以后的LGDT,EAX,CR0等关键字。

LGDT指令,不管三七二十一,把随意准备的GDT给读进来。对于这个暂定的GDT,以后还要重新设置。然后将CR0这一特殊的32位寄存器的值代入EAX,并将最高位置为0,最低位置为1,再将这个值返回给CR0寄存器。这样就完成了模式转换,进入到不用分页的保护模式。CR0,也就是control register 0,是一个非常重要的寄存器,只有操作系统才能操作它。

保护模式与先前的16位模式不同,段寄存器的解释不是16倍,而是能够使用GDT。这里的“保护”,来自英文的“protect”。在这种模式下,应用程序既不能随便改变段的设定,又不能使用操作系统专用的段。操作系统受到CPU的保护,所以称为保护模式。

在保护模式中,有带保护的16位模式,和带保护的32位模式两种。我们要使用的,是带保护的32位模式。

讲解CPU的书上会写到,通过代入CR0而切换到保护模式时,要马上执行JMP指令。所以我们也执行这一指令。为什么要执行JMP指令呢?因为变成保护模式后,机器语言的解释要发生变化。CPU为了加快指令的执行速度而使用了管道这一机制,就是说,前一条指令还在执行的时候,就开始解释下一条甚至是再下一条指令。因为模式变了,就要重新解释一遍,所以加入了JMP 指令。

而且在程序中,进入保护模式以后,段寄存器的意思也变了(不再是乘以16后再加算的意思了),除了CS以外所有段寄存器的值都从0x0000变成了0x0008。CS保持原状是因为如果CS也变了,会造成混乱,所以只有CS要放到后面再处理。0x0008,相当于“gdt + 1”的段。

\cs


\begin{code}[label=asmhead.nas节选(续)]
; bootpackの転送

		MOV		ESI,bootpack	; 転送元
		MOV		EDI,BOTPAK		; 転送先
		MOV		ECX,512*1024/4
		CALL	memcpy

; ついでにディスクデータも本来の位置へ転送

; まずはブートセクタから

		MOV		ESI,0x7c00		; 転送元
		MOV		EDI,DSKCAC		; 転送先
		MOV		ECX,512/4
		CALL	memcpy

; 残り全部

		MOV		ESI,DSKCAC0+512	; 転送元
		MOV		EDI,DSKCAC+512	; 転送先
		MOV		ECX,0
		MOV		CL,BYTE [CYLS]
		IMUL	ECX,512*18*2/4	; シリンダ数からバイト数/4に変換
		SUB		ECX,512/4		; IPLの分だけ差し引く
		CALL	memcpy
\end{code}

简单来说,这部分程序只是在调用memcpy函数。我们将这段程序写成了C语言形式。
\begin{code}
memcpy(bootpack,    BOTPAK,     512*1024/4);
memcpy(0x7c00,      DSKCAC,     512/4     );
memcpy(DSKCAC0+512, DSKCAC+512, cyls * 512*18*2/4 - 512/4);
\end{code}

函数memcpy是复制内存的函数,语法如下:

memcpy(转送源地址, 转送目的地址, 转送数据的大小);

转送数据大小是以双字为单位的,所以数据大小用字节数除以4来指定。在上面3个memcpy语句中,我们先来看看中间一句。

memcpy(0x7c00, DSKCAC, 512/4);

DSKCAC是0x00100000,所以上面这句话的意思就是从0x7c00复制512字节到0x00100000。这正好是将启动扇区复制到1MB以后的内存去的意思。下一个memcpy语句:

memcpy(DSKCAC0+512, DSKCAC+512, cyls * 512*18*2/4-512/4);

它的意思就是将始于0x00008200的磁盘内容,复制到0x00100200那里。

上文中“转送数据大小”的计算有点复杂,因为它是以柱面数来计算的,所以需要减去启动区的那一部分长度。这样始于0x00100000的内存部分,就与磁盘的内容相吻合了。


现在我们还没说明的函数就只有有程序开始处的memcpy了。bootpack是asmhead.nas的最后一个标签。haribote.sys是通过asmhead.bin和bootpack.hrb连接起来而生成的(可以通过Makefile确认),所以asmhead结束的地方,紧接着串连着bootpack.hrb最前面的部分。

memcpy(bootpack, BOTPAK, 512*1024/4);

从bootpack的地址开始的512KB内容复制到0x00280000号地址去。

这就是将bootpack.hrb复制到0x00280000号地址的处理。为什么是512KB呢?这是我们酌情考虑而决定的。内存多一些不会产生什么问题,所以这个长度要比bootpack.hrb的长度大出很多。

\cs

\begin{code}[label=asmhead.nas节选(续)]
; asmheadでしなければいけないことは全部し終わったので、
;	あとはbootpackに任せる

; bootpackの起動

		MOV		EBX,BOTPAK
		MOV		ECX,[EBX+16]
		ADD		ECX,3			; ECX += 3;
		SHR		ECX,2			; ECX /= 4;
		JZ		skip			; 転送するべきものがない
		MOV		ESI,[EBX+20]	; 転送元
		ADD		ESI,EBX
		MOV		EDI,[EBX+12]	; 転送先
		CALL	memcpy
skip:
		MOV		ESP,[EBX+12]	; スタック初期値
		JMP		DWORD 2*8:0x0000001b
\end{code}

结果我们仍然只是在做memcpy。它对bootpack.hrb的header进行解析,将执行所必需的数据传送过去。EBX里代入的是BOTPAK,所以值如下:

|[EBX + 16]|......bootpack.hrb之后的第16号地址。值是0x11a8

|[EBX + 20]|......bootpack.hrb之后的第20号地址。值是0x10c8

|[EBX + 12]|......bootpack.hrb之后的第12号地址。值是0x00310000

上面这些值,是我们通过二进制编辑器,打开harib05d的bootpack.hrb后确认的。这些值因harib的版本不同而有所变化。

SHR指令是向右移位指令,相当于“ECX $\gg$=2;”,与除以4有着相同的效果。因为二进制的数右移1位,值就变成了1/2;左移1位,值就变成了2 倍。这可能不太容易理解。还是拿我们熟悉的十进制来思考一下吧。十进制的时候,向右移动1 位,值就变成了1/10(比如120->12);向左移动1 位,值就变成了10倍(比如3 -> 30)。二进制也是一样。所以,向右移动2位,正好与除以4有着同样的效果。

JZ是条件跳转指令,根据前一个计算结果是否为0来决定是否跳转。在这里,根据SHR的结果,如果ECX变成了0,就跳转到skip那里去。在harib05d里,ECX没有变成0,所以不跳转。

而最终这个memcpy到底用来做什么事情呢?它会将bootpack.hrb第0x10c8字节开始的0x11a8字节复制到0x00310000号地址去。必须要等到“纸娃娃系统”的应用程序讲完之后才能讲清楚,以后还会说明的。

最后将0x310000代入到ESP里,然后用一个特别的JMP指令,将2$\times$8 代入到CS 里,同时移动到0x1b号地址。这里的0x1b号地址是指第2个段的0x1b 号地址。第2个段的基地址是0x280000,所以实际上是从0x28001b开始执行的。这也就是bootpack.hrb的0x1b号地址。

这样就开始执行bootpack.hrb了。

\cs

下面介绍一下“纸娃娃系统”的内存分布图。

0x00000000 - 0x000fffff : 虽然在启动中会多次使用,但之后就变空。(1MB)

0x00100000 - 0x00267fff : 用于保存软盘的内容。(1440KB)

0x00268000 - 0x0026f7ff : 空(30KB)

0x0026f800 - 0x0026ffff : IDT (2KB)

0x00270000 - 0x0027ffff : GDT (64KB)

0x00280000 - 0x002fffff : bootpack.hrb(512KB)

0x00300000 - 0x003fffff : 栈及其他(1MB)

0x00400000 - : 空

这个内存分布图其实也没有什么特别的理由,觉得这样还行,跟着感觉走就决定了。另外,虽然没有明写,但在最初的1MB范围内,还有BIOS,VRAM等内容,也就是说并不是1MB全都空着。

从软盘读出来的东西,之所以要复制到0x00100000号以后的地址,就是因为我们意识中有这个内存分布图。同样,前几天,之所以能够确定正式版的GDT和IDT的地址,也是因为这个内存分布图。

\cs

\begin{code}[label=asmhead.nas节选(续)]
waitkbdout:
		IN		 AL,0x64
		AND		 AL,0x02
        IN       AL,0x60        ; 空读(为了清空数据接收缓冲区中的垃圾数据)
		JNZ		waitkbdout		; ANDの結果が0でなければwaitkbdoutへ
		RET
\end{code}

这就是waitkbdout所完成的处理。基本上,如前面所说的那样,它与|wait_KBC_sendready|相同,但也添加了部分处理,就是从OX60号设备进行IN 的处理。也就是说,如果控制器里有键盘代码,或者是已经累积了鼠标数据,就顺便把它们读取出来。

\cs

下面是memcpy程序。

\begin{code}[label=asmhead.nas节选(续)]
memcpy:
		MOV		EAX,[ESI]
		ADD		ESI,4
		MOV		[EDI],EAX
		ADD		EDI,4
		SUB		ECX,1
		JNZ		memcpy			; 引き算した結果が0でなければmemcpyへ
		RET
\end{code}

这是复制内存的程序。

\cs

\begin{code}[label=asmhead.nas节选(续)]
		ALIGNB	16
GDT0:
		RESB	8				; ヌルセレクタ
		DW		0xffff,0x0000,0x9200,0x00cf	; 読み書き可能セグメント32bit
		DW		0xffff,0x0000,0x9a28,0x0047	; 実行可能セグメント32bit (bootpack用)

		DW		0
GDTR0:
		DW		8*3-1
		DD		GDT0

		ALIGNB	16
bootpack:
\end{code}

ALIGNB指令的意思是,一直添加DBO,直到时机合适的时候为止。什么是“时机合适”呢?大家可能有点不明白。ALIGNB 16的情况下,地址能被16 整除的时候,就称为“时机合适”。如果最初的地址能被16整除,则ALIGNB指令不作任何处理。

如果标签GDT0的地址不是8的整数倍,向段寄存器复制的MOV指令就会慢一些。所以插入了ALIGNB指令。但是如果这样,“ALIGNB 8”就够了,用“ALIGNB 16”有点过头了。最后的“bootpack:”之前,也是“时机合适”的状态,所以就适当加了一句“ALIGNB 16”。

GDT0也是一种特定的GDT。0号是空区域(null sector),不能够在那里定义段。1号和2号分别由下式设定。

|set_segmdesc(gdt + 1, 0xffffffff,   0x00000000, AR_DATA32_RW);|

|set_segmdesc(gdt + 2, LIMIT_BOTPAK, ADR_BOTPAK, AR_CODE32_ER);|
我们用纸笔事先计算了一下,然后用DW排列了出来。

GDTR0是LGDT指令,意思是通知GDT0说“有了GDT”。在GDT0里,写入了16位的段上限,和32位的段起始地址。

\cs

到此为止,关于asmhead.nas的说明就结束了。就是说,最初状态时,GDT在asmhead.nas里,并不在0x00270000 ~ 0x0027ffff的范围里。IDT连设定都没设定,所以仍处于中断禁止的状态。应当趁着硬件上积累过多数据而产生误动作之前,尽快开放中断,接收数据。

因此,在bootpack.c的HariMain里,应该在进行调色板(palette)的初始化以及画面的准备之前,先赶紧重新创建GDT和IDT,初始化PIC,并执行“|io_sti();|”。

\begin{code}[label=bootpack.c节选]
void HariMain(void)
{
    struct BOOTINFO *binfo = (struct BOOTINFO *) ADR_BOOTINFO;
    char s[40], mcursor[256], keybuf[32], mousebuf[128];
    int mx, my, i;
    struct MOUSE_DEC mdec;

    init_gdtidt();
    init_pic();
    io_sti(); /* IDT/PICの初期化が終わったのでCPUの割り込み禁止を解除 */
	fifo8_init(&keyfifo, 32, keybuf);
	fifo8_init(&mousefifo, 128, mousebuf);
	io_out8(PIC0_IMR, 0xf9); /* PIC1とキーボードを許可(11111001) */
	io_out8(PIC1_IMR, 0xef); /* マウスを許可(11101111) */

    init_keyboard();

    init_palette();
    init_screen8(binfo->vram, binfo->scrnx, binfo->scrny);
\end{code}
